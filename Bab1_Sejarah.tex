\chapter{Pengenalan Linux}

\section{Sejarah Singkat Linux}
(Sejarah unix dan Unix War) Awal dari sistem operasi modern dimulai pada tahun 1969. Pada tahun itu, bahasa pemrograman C dan sistem operasi Unix diciptakan oleh Dennis Ritchie dan Ken Thomson di At\&T Lab. \\ 

(Sejarah GNU) \\ 

(Tercipatnya Linux) \\ 

Saat ini, Linux merupakan salah satu sistem operasi dengan pengguna yang paling banyak. Sekitar 3\% komputer rumahan dan lebih dari 97\% komputer super terkemuka di dunia menggunakan Linux. Mayoritas sever juga menggunakan Linux. Kendati demikian, kepopuleran Linux khususnya pada orang umum jika dibandingkan dengan sistem operasi Windows masih sangat jauh. Hal ini dikarenakan kebanyakan orang umum menganggap Linux itu sulit dan segala hal menggunakan \textit{Command-Line Interface} (CLI). Namun seiring perkembangan waktu, persepsi ini sudah dapat diubah dengan adanya \textit{Desktop Environment} seperti GNOME, KDE, dan lainnya yang menyediakan GUI yang sangat ramah bagi pengguna komputer rumahan.\\

\section{Linux Distro}
Linux bersifat \textit{open-source}. Karena sifatnya itu, sangat banyak variasi Linux atau yang biasa juga disebut distro (\textit{distribution}) yang tersebar di internet. Distro merupakan kumpulan software yang dibangun di atas kernel Linux. Linux distro biasanya terdiri dari \textit{desktop environment}, \textit{pre-installed software}, dan \textit{package manager} (dnf, apt-get, pacman, dan lain-lain). Setiap distro juga memiliki filosofi adn tujuannya masing-masing. Misalnya Ubuntu yang fokus pada kenyamanan pengguna dengan menyediakan desktop yang lengkap, berbeda dengan distro Arch yang lebih fokus pada \textit{simplicity} dan memberikan penggunanya keleluasan untuk menginstall \textit{package} yang hanya mereka butuhkan. Keduanya tentu memiliki kelebihan dan kekurangan masing-masing. 

\subsection{Ubuntu}
